\documentclass[linearalgebra]{subfiles}

\begin{document}

    \chap{Linear Equations} 

    \section{Elementary Matrix Operations and Elementary Matrices}

    \begin{definition}{Elementary Operations}{on Matrix}
        Let $A\in M_{m\times n}(\F)$. We define the three \emph{elementary row operations} on $A$ as follows.

        \begin{enumerate}
            \item Interchange any two rows of $A$.
            \item Multiply any row of $A$ by a nonzero scalar $c\in \F$.
            \item Add a scalar multiple of any row of $A$ to another row of $A$.
        \end{enumerate}
        We say (a) is a \emph{type 1} operation, (b) is a \emph{type 2} operation, and (c) is a \emph{type 3} operation.
    \end{definition}

    \begin{remark}
        We have an analogous definition for the three elementary column operations.
    \end{remark}

    \begin{definition}{Elementary Matrix}{}
        We say a matrix $E\in M_{n\times n}(\F)$ is an \emph{elementary matrix} if $E$ is obtained by applying any elementary operation on $I$. Depending on which type of operation is used, we call that $E$ is a \emph{type 1}, \emph{type 2}, or \emph{type 3} elementary matrix, respectively.
    \end{definition}
    
    \begin{remark}
        Any elementary matrix can be obtained by at least two ways. That is, if $E\in M_{n\times n}(\F)$ is obtained by applying an elementary row operation on $I$, then there is an elementary column operation of the same type by applying which $E$ is obtained from $I$.
    \end{remark}

    \begin{theorem}{}
        Let $A\in M_{m\times n}(\F)$, and suppose that $B$ is obtained from $A$ by performing an elementary column [row] operation on $A$. Then there exists an $n\times n$ [$m\times m$] elementary matrix $E$ such that $B=AE$ [$B=EA$]. In fact, $E$ is obtained from $I$ by performing the same elementary column [row] operation. Conversely, for any $n\times n$ [$m\times m$] elementary matrix $E$, $B=AE$ [$B=EA$] is the matrix obtained by performing an elementary column [row]
        operation on $I$ to obtain $E$.
    \end{theorem}

    \begin{prop}{Invertibility of Elementary Matrices}
        Let $E\in M_{n\times n}(\F)$ be elementary. Then $E$ is invertible and $E^{-1}$ is an elementary matrix of the same type.
    \end{prop}

    \begin{proof}
        Let $E\in M_{n\times n}(\F)$ be invertible. Then $E$ can be obtained from $I$ by performing elementary operations, so $I$ can be obtained from $E$ by reversing the operations. Then by Theorem 3.1, there exists an $E'\in M_{n\times n}(\F)$ such that $EE'=E'E=I$. Therefore $E$ is invertible and $E' = E^{-1}$.
    \end{proof}

    \section{The Rank and Inverse of a Matrix}
    
    \begin{definition}{Rank}{of a Matrix}
        Let $A\in M_{n\times n}(\F)$. We define the \emph{rank} of $A$, denoted by $\rank(A)$, to be the rank of the left multiplication transformation $L_A: \F^n\to \F^m$.
    \end{definition}

    \begin{theorem}{Invertible Matrix Theorem I}
        Let $A\in M_{m\times n}(\F)$, and let $P\in M_{m\times m}(\F), Q\in M_{n\times n}(\F)$ be invertible. 
        \begin{enumerate}
            \item $\rank(AQ) = \rank(A)$. 
            \item $\rank(PA) = \rank(A)$. 
            \item $\rank(PAQ) = \rank(A)$.
        \end{enumerate}
    \end{theorem}
    
    \begin{proof}
        For (a), observe that 
        \begin{equation*}
            \rank(AQ) = \rank(L_{AQ}) = \rank(L_AL_Q) = \dim (L_AL_Q(\F^n)) = \dim(L_A(\F^n)) = \rank(A),
        \end{equation*}
        since $Q$ is an invertible matrix so $L_Q$ is an isomorphism and $L_Q(\F^n) = \image(L_Q) = \F^n$. Similar argument can be used for (b). Observe that (c) is an immediate consequence of (a) and (b).
    \end{proof}

    \begin{cor}{}
        Let $A\in M_{m\times n}(\F)$. Then, for all elementary $E\in M_{n\times n}, F\in M_{m\times m}$,
        \begin{equation*}
            \rank(A) = \rank(AE) = \rank(AF).
        \end{equation*}
    \end{cor}	

    \begin{theorem}{Invertible Matrix Theorem II}
        Let $A\in M_{n\times n}(\F)$. Then $A$ is invertible if and only if $\rank(A) = n$.
    \end{theorem}
    
    \begin{proof}
        For the forward direction, suppose that $A$ is invertible. Then $L_A$ is an isomorphism, so $\rank(A) = \rank(L_A) = \dim (\F^n) = n$. For the reverse direction, suppose $\rank (A) = n$ then $\rank (A) = n = \dim( \F^n)$ so $L_A: \F^n\to \F^n$ is an isomorpshism, and thus $A$ is invertible.
    \end{proof}

    \begin{prop}{Rank of a Matrix Is the Maximal Number of Independent Columns}
        Let $A\in M_{m\times n}(\F)$. Then $\rank(A)$ is equal to the maximal number of independent columns.
    \end{prop}

    \begin{proof}
        Write $A = [A_1\ A_2\ \cdots \ A_n]$ and let $\alpha = \left\lbrace A_1, A_2, \ldots, A_n \right\rbrace$, the set of columns of $A$. Moreover, let $\beta = \left\lbrace e_1, e_2, \ldots, e_n \right\rbrace$ be the standard basis for $\F^n$. Then
        \begin{align*}
            \rank(A) & = \rank(L_A) = \dim (\image(L_A)) = \dim (L_A(\F^n)) = \dim \left(\spn \left(L_A(\beta)\right)\right) \\
                    & = \dim \left( \spn \left\lbrace Ae_1, Ae_2, \ldots, Ae_n \right\rbrace \right) = \dim \left( \spn \left\lbrace A_1, A_2, \ldots, A_n \right\rbrace \right) = \dim ( \spn(\alpha) ). 
        \end{align*}
        But $\dim(\spn(\alpha))$ is the maximal number of independent vectors of $\alpha$, which is the desired result.
    \end{proof}

    \begin{remark}
        An important restatement of Proposition 3.5 is that, for any $A\in M_{m\times n}(\F)$, $\rank(A)$ is the dimension of the subspace of $\F^m$ that the columns of $A$ span.
    \end{remark}

    \begin{theorem}{Matrix Elimination}
            Let $A=M_{m\times n}(\F)$. Then by finite number of elementary operations $E_1, E_2, \ldots, E_p$ on $A$, $A$ can be transformed into
            \begin{equation*}
                E_pE_{p-1}\cdots A\cdots E_2E_1 = \begin{bmatrix} I_r & O_1 \\ O_2 & O_3 \end{bmatrix}, 
            \end{equation*}
            where $O_1, O_2, O_3$ are zero matrices and $0\leq r = \rank(A) \leq \min(m,n)$.
    \end{theorem}

    \begin{proof}
        Consider spliting the theorem into two cases. First suppose that $A=0$ then the proof is done. So suppose that $A\neq 0$. Then there must be a nonzero entry, by type 1 operations which can be moved to $(1,1)$ position. Then, by a type 2 operation, it can be turned into $1$, and all $(n,1)$ entries can be made zero by type 3 operations. Observe that we have made a matrix of the form
        \begin{equation*}
            \begin{bmatrix} I_1 & O \\ O & A' \end{bmatrix}.
        \end{equation*}
        So by proceeding on $A'$ inductively, at most $\min(m, n)$ times, we have the desired result. Moreover, since the number of inductive process is finite, we also observe that the number of elementary operations that are used to obtain a matrix of the desired form is finite. Lastly,
        \begin{equation*}
            \rank(A) = \rank(I_r) = r 
        \end{equation*}
        by Lemma 3.5 and invertible matrix theorem I, so $0\leq r = \rank(A) \leq \min(m, n)$.
    \end{proof}

    \begin{cor}{}
        For any $A\in M_{m\times n}(\F)$, there exist invertible $P\in M_{m\times m}(\F)$ and $Q\in M_{n\times n}(\F)$ such that $PAQ$ is of the form
        \begin{equation*}
            \begin{bmatrix}
                I_r & O \\ O & O
            \end{bmatrix}.
        \end{equation*}
    \end{cor}	

    \begin{cor}{}
        Let $A\in M_{m\times n}(\F)$. 
        \begin{enumerate}
            \item $\rank(A) = \rank\left( A^{T}  \right) $.
            \item $\rank(A)$ is the number of maximal independent rows.
            \item Columns and rows of $A$ span subspaces of $\F^m$ of equal dimension.
        \end{enumerate}
    \end{cor}	

    \begin{proof}
        For (a), let $P\in M_{m\times m}(\F)$ and $Q\in M_{n\times n}$ be invertible matrices discussed in Corollary 3.6.1. Then,
        \begin{equation*}
            \rank(A) = \rank(PAQ) = \rank \left( (PAQ)^T \right) = \rank \left( Q^TA^TP^T \right) = \rank (A^T). 
        \end{equation*}
        Observe that (b) and (c) are immediate consequences of (a) and Lemma 3.5.
    \end{proof}

    \begin{cor}{An Invertible Matrix Is a Product of Elementary Matrices}
        Let $A\in M_{n\times n}(\F)$ be invertible. Then $A$ is a product of elementary matrices.
    \end{cor}	

    \begin{proof}
        Since $A$ is invertible, by invertible matrix theorem II, $\rank A = n$, and by Theorem 3.6 and Corollary 3.6.1, there exist invertible $P=\prod^p_{i=1}E_i, Q=\prod^q_{j_1}F_i\in M_{n\times n}(\F)$ such that
        \begin{equation*}
            PAQ = I.
        \end{equation*}
        That is,
        \begin{equation*}
            A = P^{-1}IQ^{-1} = P^{-1}Q^{-1} = E_p^{-1}E_{p-1}^{-1}\cdots E_2^{-1}E_1^{-1} F_q^{-1}F_{q-1}^{-1}\cdots F_2^{-1}F_1^{-1},
        \end{equation*}
        where each $E_i^{-1}$ and $F_i^{-1}$ are elementary matrices by Proposition 3.2.
    \end{proof}

    \begin{prop}{}
        Let $A\in M_{m\times n}(\F)$ and $B\in M_{n\times p}(\F)$. Then
        \begin{equation*}
            \rank(AB) = \min(\rank(A), \rank(B)).
        \end{equation*}
    \end{prop}

    \begin{proof}
        First, suppose $\rank (A)\leq \rank(B)$. Observe that
        \begin{equation*}
            \rank(AB) = \rank( L_{AB}) = \rank (L_AL_B) = \dim (L_A(\image(L_B)))
        \end{equation*}
        where $\image(L_B)\subseteq \F^n$ so $L_A(\image(L_B))\subseteq L_A(\F^n)$. Thus
        \begin{equation*}
            \dim (L_A(\image(L_B))) \leq \dim (L_A(\F^n)) = \dim (\image(L_A)) = \rank(A).
        \end{equation*}
        When $\rank(B)\leq \rank(A)$, the same argument can be used by taking the transpose of $AB$, $B^TA^T$.
    \end{proof}

    \section{Four Fundamental Subspaces of a Matrix}

    \begin{definition}{Column Space, Row Space, Null Space, Left Null Space}{of a Matrix}
        Let $A\in M_{m\times n}(\F)$. Define the \emph{column space} of $A$, $\image(A)$, by
            \begin{equation*}
                \image(A) = \spn \left\lbrace x\in \F^m: x\text{ column of }A \right\rbrace \subseteq\F^m.
            \end{equation*}
            Similarly, define the \emph{row space} of $A$, $\image\left( A^{T}  \right) $, by
            \begin{equation*}
                \image\left( A^{T}  \right)  = \spn \left\lbrace x\in \F^n: x\text{ row of }A \right\rbrace\subseteq\F^n.
            \end{equation*}
            Moreover, define the \emph{null space} of $A$, $\ker(A)$, by
            \begin{equation*}
                \ker(A) = \left\lbrace x\in \F^n: Ax = 0 \right\rbrace\subseteq\F^n.
            \end{equation*}
            Lastly, define the \emph{left null space} of $A$, $\ker\left( A^{T}  \right) $, by
            \begin{equation*}
                \ker (A^T) = \left\lbrace x\in \F^m: A^Tx = 0 \right\rbrace\subseteq\F^m.
            \end{equation*}
            $\image(A), \image\left( A^{T} \right) , \ker(A), \ker\left( A^{T}  \right)$ together are called the \emph{four fundamental subspaces} of $A$.
    \end{definition}

    \begin{prop}{Properties of Four Fundamental Subspaces}
        Let $A\in M_{m\times n}(\F)$. Then the following holds.
        \begin{enumerate}
            \item $\image(A)$ and $\ker(A^T)$ are subspaces of $\F^m$ and $\image\left( A^{T}  \right) $ and $\ker(A)$ are subspaces of $\F^n$.
            \item $\rank(A) = \dim (\image\left( A^{T}  \right) ) = \dim (\image(A))$.
            \item $\dim (\image(A)) + \dim (\ker( A^T)) = m$ and $\dim (\image\left( A^{T}  \right) ) + \dim (\ker(A)) = n$.
            \item $\image(A) \oplus \ker(A^T) = \F^m$ and $\image\left( A^{T} \right)  \oplus \ker(A) = \F^n$.
        \end{enumerate}
    \end{prop}

    \begin{proof}
        For (a), use Proposition 1.7 for each set. For (b), observe that
        \begin{align*}
            \rank(A) & = \text{the number of linearly independent columns of }A \\
                    & = \dim (\spn \left\lbrace \col_1(A), \ldots, \col_n(A) \right\rbrace) = \dim (\col (A)),
        \end{align*} 
        and similar proof holds for $\rank(A) = \dim \left( \image\left( A^{T}  \right)  \right) $. For (c), observe that
        \begin{equation*}
            \dim (\F^n) = \dim (\image(L_A)) + \dim (\ker(L_A)) = \rank(A) + \dim (\ker (A)) = \dim \left( \image\left( A^{T} \right)  \right)  + \dim (\ker(A)),
        \end{equation*}
        by rank-nullity theorem, and similar proof holds for the remaining part of the statement. For (d), observe that
        \begin{equation*}
            \image\left( A^{T} \right)  \cap \ker(A) = \left\lbrace 0 \right\rbrace,
        \end{equation*}
        so $\image\left( A^{T} \right) \oplus \ker(A)$ is well defined. Then we have
        \begin{equation*}
            \dim \left(\image\left( A^{T}  \right)  \oplus \ker(A)\right) = \dim \left( \image\left( A^{T} \right)  \right)  + \dim (\ker(A)) - \dim (\image\left( A^{T}  \right)  \cup \ker(A)) = n,
        \end{equation*}
        where $\image\left( A^{T} \right)  \oplus \ker(A)\subseteq \F^n$, so $\image\left( A^{T} \right)  \oplus \ker(A) = \F^n$. Again, similar proof holds for the remaining part of the statement.
    \end{proof}

    \clearpage
    \begin{theorem}{Invertible Matrix Theorem III}
        Let $A\in M_{n\times n}(\F)$. Then the following are equivalent.
        \begin{enumerate}
            \item $A$ is invertible.
            \item Columns of $A$ form a basis for $\F^n$.
            \item Rows of $A$ form a basis for $\F^n$.
            \item $A$ is a product of elementary matrices.
        \end{enumerate}
    \end{theorem}

    \begin{proof}
        Consider showing that (a) is equivalent to other statements. For (a)$\iff$(b), observe that
        \begin{align*}
            A\text{ is invertible} & \iff \rank(A) = n \\
                                   & \iff \text{maximal number of independent columns is }n \\
                                   & \iff \text{columns of }A\text{ form a basis for }\F^n.
        \end{align*}
        Observe that similar proof holds for (a)$\iff$(c). (a)$\implies$(d) is provided by Corollary 3.6.3. (d)$\implies$(a) is also clear, since each elementary matrix is invertible. 
    \end{proof}

    \begin{definition}{Augmented}{Matrix}
        A matrix of the form $\left[ A\mid B \right]$ for some $A\in M_{m\times n}(\F)$ and $B\in M_{m\times p}(\F)$ is called an \emph{augmented matrix}.  
    \end{definition}

    \begin{prop}{Computing the Inverse Matrix}
        Let $A\in M_{n\times n}(\F)$ be invertible. Then the following hold.
        \begin{enumerate}
            \item There exists finite number of row operations that transforms $[A\mid I]$ into $[I\mid A^{-1}]$.
            \item If there exists $B\in M_{n\times n}(\F)$ such that $[I\mid B]$ is obtained from $[A\mid I]$ by a finite number of row operations, then $A$ is invertible and $B = A^{-1}$.
        \end{enumerate}
    \end{prop}

    \begin{proof}
        For (a), write $C = [A\mid I] = \begin{bmatrix} C_1 & C_2 & \cdots & C_m \end{bmatrix}$ for convenience. Then for any $B\in M_{n\times n}(\F)$,
        \begin{equation*}
            BC = 
            \begin{bmatrix}
                BC_1 & BC_2 & \cdots & BC_m
            \end{bmatrix},
        \end{equation*}
        so
        \begin{equation*}
            A^{-1}[A\mid I] = [I\mid A^{-1}],
        \end{equation*}
        where $A^{-1} = E_1E_2\cdots E_p$ for some elementary matrices $E_1, \ldots, E_p\in M_{n\times n}(\F)$ by Theorem 3.9. That is,
        \begin{equation*}
            E_1\cdots E_p [A\mid I] = [I\mid A^{-1}]
        \end{equation*}
        which means there exist corresponding row operations to $E_1, \ldots, E_p$ that transform $[A\mid I]$ into $[I\mid A^{-1}]$. For (b), observe that there are elementary matrices $G_1, \ldots, G_q$ corresponding to row operations that transform $[A\mid I]$ into $[I\mid B]$. That is,
        \begin{equation*}
            G_1\cdots G_q [A\mid I] = [G_1\cdots G_qA\mid G_1\cdots G_qI] = [I\mid B],
        \end{equation*}
        so $G_1\cdots G_q = A^{-1}$ and thus $B = G_1\cdots G_q = A^{-1}$, as desired.
    \end{proof}

    \section{Systems of Linear Equations}

    \begin{definition}{System of Linear Equations}{}
        A collection of linear equations of the form
        \begin{equation*}
            \begin{cases}
                a_{11}x_1 + a_{12}x_2 + \cdots + a_{1n}x_n & = b_1 \\
                a_{21}x_1 + a_{22}x_2 + \cdots + a_{2n}x_n & = b_2 \\
                                                           & \vdots \\
                a_{m1}x_1 + a_{m2}x_2 + \cdots + a_{mn}x_n & = b_m
            \end{cases},
        \end{equation*}
        where $a_{ij}, b_i\in \F$ for all $i\in \left\lbrace 1, \ldots, m \right\rbrace, j\in \left\lbrace 1, \ldots, n \right\rbrace$ and $x_1, \ldots, x_n$ are $n$ variables taking values in $\F$, is called a \emph{system of linear equations} in $n$ unknowns over $\F$.
    \end{definition}

    \begin{definition}{Coefficient Matrix, Augmented Matrix, Solution, Solution Set}{of a System}
        Any system can be written as a matrix product $Ax = b$, where
        \begin{equation*}
            Ax = 
            \begin{bmatrix}
                a_{11} & a_{12} & \cdots & A_{1n} \\ a_{21} & a_{22} & \cdots & a_{2n} \\ \vdots & \vdots & & \vdots \\ a_{m1} & a_{m2} & \cdots & a_{mn}
            \end{bmatrix} \ \ 
            \begin{bmatrix}
                x_1 \\ x_2 \\ \vdots \\ x_n
            \end{bmatrix} \ \ 
            =
            \begin{bmatrix}
                b_1 \\ b_2 \\ \vdots \\ b_m
            \end{bmatrix}
            = b.
        \end{equation*}
        Moreover, the matrix $A$ and $[A\mid b]$ are called the \emph{coefficient matrix} and \emph{augmented matrix} of the system, respectively. We say $c\in \F^n$ is a \emph{solution} of the system if it satisfies $Ac=b$. The set of all solutions $\left\lbrace c\in \F^n: Ac=b \right\rbrace$ is called the \emph{solution set} of the system.
    \end{definition}

    \begin{definition}{Consistent, Inconsistent, Homogeneous, Inhomogeneous}{System}
        A system $Ax= b$ is called \emph{consistent} if its solution set is nonempty and \emph{inconsistent} otherwise. Moreover, it is called \emph{homogeneous} if $b=0$ and \emph{inhomogeneous} otherwise.
    \end{definition}

    \begin{prop}{The Solution Set of a Homogeneous System Is a Subspace}
        Let $A\in M_{m\times n}(\F)$ and consider $Ax=0$. Then the solution set $K_H$ of the system is a subspace of $\F^n$ and
        \begin{equation*}
            \dim\left( K_H \right)  = n - \rank(A).
        \end{equation*}
    \end{prop}

    \begin{proof}
        Observe that $A0 = 0$ so $0\in K_H$. For closure under subtraction, let $v, u\in K_H$ then
        \begin{equation*}
            A(v-u) = Av-Au = 0-0 = 0
        \end{equation*}
        so $(v-u)\in K_H$. For closure under multiplication, let $c\in \F$ then
        \begin{equation*}
            A(cv) = c(Av) = c0 = 0
        \end{equation*}
        so $cv\in K_H$. Moreover, observe that $K_H = \ker(L_A)$, so by the rank-nullity theorem
        \begin{equation*}
            n = \dim(\F^n) = \nullity (L_A) + \rank (L_A) = \dim (\ker(L_A)) + \rank(A) = \dim(K_H) + \rank(A)
        \end{equation*}
        rearranging which in terms of $\dim(K_H)$ gives
        \begin{equation*}
            \dim(K_H) = n - \rank(A). \eqedsym
        \end{equation*}
    \end{proof}

    \clearpage
    \begin{cor}{Properties of the Solution Set of a Homogeneous System}
        Let $K_H$ be the solution set to $Ax=0$. Then the following hold.
        \begin{enumerate}
            \item $K_H\neq \emptyset$. In particular, $0\in K_H$.
            \item $K_H = \left\lbrace 0 \right\rbrace \iff \rank(A) = n$.
            \item If $m<n$, then the system has a nonzero solution.
        \end{enumerate}
    \end{cor}	

    \begin{definition}{Full Column Rank}{of a System}
            We say a system $Ax=b$ is of \emph{full column rank} if it satisfies (b) of Corollary 3.11.1. 
    \end{definition}

    \begin{prop}{$K=K_H + \left\lbrace c \right\rbrace$ for any Particular Solution $c$}
        Suppose $A\in M_{m\times n}(\F)$ and $b\in \F^m$. Let $K$ and $K_H$ be the solution sets for $Ax=b$ and $Ax=0$, respectively. Then for any solution $c$ to $Ax=b$, we have
        \begin{equation*}
            K = \left\lbrace c \right\rbrace + K_H = \left\lbrace c+k: k\in K_H \right\rbrace.
        \end{equation*}
    \end{prop}

    \begin{proof}
        Let $x\in \F^n$ such that $Ax=b$. Then for any $k\in K_H$, observe that
        \begin{equation*}
            A(x+k) = Ax+Ak = Ax + 0 = Ax = b,
        \end{equation*}
        so $(x+k)\in K$. Moreover, if $z\in \F^n$ such that $Az = b$, then
        \begin{equation*}
            A(x-z) = b-b = 0
        \end{equation*}
        so $(x-z)\in K_H$. In other words, there exists $w = (x-z)\in K_H$ such that
        \begin{equation*}
            x = z + w. \eqedsym
        \end{equation*}
    \end{proof}

    \begin{theorem}{Invertible Matrix Theorem IV}
        Let $A\in M_{n\times n}(\F)$. Then $A$ is invertible if and only if $Ax=b$ has a unique solution.
    \end{theorem}

    \begin{proof}
        For the forward direction, suppose that $A$ is invertible. From $Ax=b$, we have
        \begin{equation*}
            A^{-1}b = A^{-1}Ax = Ix = x,
        \end{equation*}
        so $Ax=b$ has a unique solution $x=A^{-1}b$. For the reverse direction, suppose $Ax=b$ has a unique solution $c\in \F^n$. Let $K_H$ and $K$ be the solution sets for $Ax=0$ and $Ax=b$, respectively. By Proposition 3.12,
        \begin{equation*}
            K = \left\lbrace c \right\rbrace = \left\lbrace c \right\rbrace + K_H
        \end{equation*}
        so $K_H = \left\lbrace 0 \right\rbrace$ and $\dim(K_H) = 0$. Since
        \begin{equation*}
            \dim(K_H) = n - \rank(A),
        \end{equation*}
        it follows that $\rank(A) = n$, which means $A$ is invertible.
    \end{proof}

    \begin{prop}{A System Is Consistent If and Only If $\rank(A) = \rank [A\mid b]$}
        Let $Ax = b$ be a system of linear equations. Then the system is consistent if and only if $\rank(A) = \rank[A\mid b]$.
    \end{prop}

    \begin{proof}
        Observe that
        \begin{align*}
            & Ax=b\text{ has a solution} \\ 
            & \iff b\in \image(L_A) \\
            & \iff b\in \spn \left\lbrace \col_1(A), \ldots, \col_n(A) \right\rbrace \\
            & \iff \spn \left\lbrace \col_1(A), \ldots, \col_n(A) \right\rbrace = \spn \left\lbrace \col_1(A), \ldots, \col_n(A), b \right\rbrace \\
            & \iff \dim \left( \spn \left\lbrace \col_1(A), \ldots, \col_n(A) \right\rbrace \right) = \dim \left( \spn \left\lbrace \col_1(A), \ldots, \col_n(A), b \right\rbrace \right) \\
            & \iff \rank(A) = \rank [A\mid b],
        \end{align*}
        which is the desired result.
    \end{proof}

    \begin{definition}{Equivalent}{Systems}
        Two systems of linear equations are called \emph{equivalent} if they have the same solution set.
    \end{definition}

    \begin{prop}{$CAx=Cb$ Is Equivalent to $Ax=b$ Whenever $C$ Is Invertible}
            Let $Ax=b$ be a system of $m$ linear equations in $n$ unknowns and let $C\in M_{m\times n}(\F)$ be invertible. Then the system $(CA)x = Cb$ is equivalent to $Ax = b$.
    \end{prop}

    \begin{proof}
        Let $K_H$ be the solution set of $Ax=b$ and suppose that $x\in K_H$. Then
        \begin{equation*}
            (CA)x = C(Ax) = Cb,
        \end{equation*}
        so $x$ is in the solution set of $CAx=Cb$ as well. Moreover, suppose that $x$ is in the solution set of $(CA)x = Cb$, then
        \begin{equation*}
            CAx = Cb \iff C^{-1}CAx = C^{-1}Cb \iff Ax = b.
        \end{equation*}
        Thus $x\in K_H$, which is the desired result.
    \end{proof}

    \begin{cor}{}
        Let $Ax=b$ be a system of $m$ linear equations in $n$ unknowns. If $[A'\mid b']$ is obtained from $[A\mid b]$ by a finite number of elementary row operations, then the system $A'x=b'$ is equivalent to $Ax=b$.
    \end{cor}	

    \begin{definition}{Reduced Row Echelon Form (RREF), Leading One}{}
            A matrix is said to be in \emph{reduced row echelon form} if the following three conditions are satisfied.
            \begin{enumerate}
                \item Any row containing nonzero entry precedes any row in which all the entries are zero, if any.
                \item The first nonzero entry in each row is the only nonzero entry in its column.
                \item The first nonzero entry in each row is $1$, called the \emph{leading one} of the row, and it occurs a column to the right of the first nonzero entry in the preceding row.
            \end{enumerate}
    \end{definition}

    \begin{remark}
        Given $A\in M_{m\times n}(\F)$, we may obtain an RREF from $A$ as follows. 
        \begin{enumerate}
            \item In the leftmost nonzero column, use elementary row operations to get $1$ in the first row.
            \item By means of type 3 elementary row operations, use the first row to obtain zeroes in the remaining entries of the leftmost nonzero column.
            \item Consider the submatrix consisting of the columns to the right of the column we just modified and the rows beneath the row that just got a leading one. Use elementary row operations - if necessary - to get a leading one in the top of the first nonzero column of this submatrix.
            \item Use elementary row operations to obtain zeroes below the one created in the preceding step.
            \item Repeat (c) and (d) until no nonzero rows remain.
            \item Work upward, beginning with the last nonzero row and add multiples of each rows above to create zeroes above the first nonzero in each row.
            \item Repeat the precess in (f) for each preceding row until it is performed with the second row, at which time the reduction process is complete.
        \end{enumerate}
    \end{remark}

    \begin{definition}{Gaussian Elimination}{}
        We call the procedure described in Remark 3.4 the \emph{Gaussian elimination}.
    \end{definition}

    \begin{prop}{Gaussian Elimination}
        Gaussian elimination transforms any matrix to its RREF, and the RREF of a matrix is unique.
    \end{prop}

    \begin{definition}{Free Variable}{}
        Let $R$ be the RREF of a coefficient matrix of a system of linear equations $Ax=b$. If the $j$th column of $R$ does not contain a leading one, then $x_j$ is called a \emph{free variable}.
    \end{definition}

    \begin{prop}{Number of Free Variables Is Equal to $n-\rank(A)$}
        Let $A\in M_{m\times n}(\F), b\in \F$, and $B$ be the RREF of $A$. Then the following holds.
        \begin{equation*}
            \text{number of free variables} = n - \text{number of leading ones} = n - \rank(A). 
        \end{equation*}
    \end{prop}

    \begin{proof}
        Observe that
        \begin{equation*} 
            \rank(A) = \rank(B) = \text{number of reading ones of } B = \text{number of nonzero rows of } B. \eqedsym 
        \end{equation*}
    \end{proof}

    \begin{remark}
        Let $A\in M_{m\times n}(\F)$ and $b\in \F$. Consider solving the system $Ax=b$ by the following algorithm.
        \begin{enumerate}
            \item Write the augmented matrix $[A\mid b]$ for the system.
            \item Use elementary row operations (e.g. Gaussian elimination) to transform the augmented matrix into its RREF $[A'\mid b']$.
            \item Write the system of linear equations to the RREF.
            \item If the system contain an equation of the form $0=1$, then the system is inconsistent.
            \item Otherwise, assign values $t_1, \ldots, t_{n-r}$ to the free variables and then solve the remaining variables in terms of the free variables. Here $r = \rank(A') = \rank(A)$ is the number of nonzero rows of $A'$.
            \item Then an arbitrary solution to $Ax=b$ is of the form
                \begin{equation*}
                    x = x_0 + \sum^{n-r}_{i=1} t_iu_i
                \end{equation*}
                where $r$ is the number of nonzero rows in $A'$.
        \end{enumerate}
        Then $K$ is given by
        \begin{equation*}
            K = \left\lbrace x\in \F^n: x = x_0 + \sum^{n-r}_{i=1} t_iu_i \right\rbrace. 
        \end{equation*}
    \end{remark}

    \begin{definition}{Parametric Value}{Assigned to Free Variables}
        The values $t_1, \ldots, t_{n-r}$ assigned to free variables in (e) of Remark 3.4 are called \emph{parametric values}.
    \end{definition}

    \begin{prop}{}
        Let $[A\mid b]$ be a consistent system of $m$ linear equations in $n$ variables. Suppose that the RREF of $[A\mid b]$ has $r$ nonzero rows. If the general solution to $Ax=b$ obtained by the procedure described in Remark 3.4 is of the form
        \begin{equation*}
            x = x_0 + \sum^{n-r}_{i=1} t_iu_i,
        \end{equation*}
        then $x_0\in \F^n$ is a solution to $Ax=b$ and $\left\lbrace u_1, \ldots, u_{n-r} \right\rbrace$ is a basis for the solution set of the corresponding homogeneous system $Ax = 0$.
    \end{prop}

    \begin{proof}
        To verify that $x_0$ is a solution to $Ax=b$, observe that, by setting $t_1, \ldots, t_{n-r} = 0$,
        \begin{equation*}
            x = x_0 + \sum^{n-r}_{i=1} 0u_i = x_0.
        \end{equation*}
        So, we may verify that $\left\lbrace u_1, \ldots, u_{n-r} \right\rbrace$ generates $K_H$ the following. First, write
        \begin{equation*}
            K = \left\lbrace x_0 \right\rbrace + K_H,
        \end{equation*}
        where $K_H$ and $K$ are solution sets of $Ax=0$ and $Ax=b$, respectively. Observe that
        \begin{equation*}
            K = \left\lbrace x: x + x_0 + \sum^{n-r}_{i=1} t_iu_i \right\rbrace = \left\lbrace x_0 \right\rbrace + \left\lbrace x: x=\sum^{n-r}_{i=1} t_iu_i \right\rbrace,
        \end{equation*}
        so it must be the case which $K_H = \left\lbrace x: x=\sum^{n-r}_{i=1} t_iu_i \right\rbrace$. But this means $K_H = \spn \left\lbrace u_1, \ldots, u_{n-r} \right\rbrace$. Lastly, to verify that $\left\lbrace u_1, \ldots, u_{n-r} \right\rbrace$ is linearly independent, observe that
        \begin{equation*}
            \dim\left( K_H \right)  = n-\rank(A) = n-r
        \end{equation*}
        by Proposition 3.11.
    \end{proof}

\end{document}
