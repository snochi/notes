\documentclass[linearalgebra]{subfiles}

\begin{document}

    \chap{Polynomials}

    \section{Algebras} 

    \begin{definition}{Algebra}{over a Field}
        We say $\alg$ is \emph{algebra} over a field $\F$ if $\alg$ is a vector space over $\F$ equipped with a binary operation $\alg\times\alg\to\alg$ called multiplication that has the following properties. Let $\alpha, \beta, \gamma\in \alg$ and $c\in \F$. 

        \begin{enumerate}
            \item Multiplication is distributive with respect to addition.
                \begin{equation*}
                    \alpha(\beta+\gamma) = \alpha\beta + \alpha\gamma \ \ \ \ (\beta+\gamma)\alpha = \beta\alpha + \gamma\alpha.
                \end{equation*}
            \item Multiplication is compatible with respect to scalar multiplication.
                \begin{equation*}
                    c(\alpha\beta) = (c\alpha)\beta = \alpha(c\beta).
                \end{equation*}
        \end{enumerate}
    \end{definition}

    \begin{definition}{Associative, Commutative}{Algebra}
        We say an algebra $\alg$ is \emph{associative} if the multiplication on $\alg$ is associative. That is, for any $\alpha,\beta,\gamma\in\alg$,
        \begin{equation*}
            \alpha(\beta\gamma) = (\alpha\beta)\gamma.
        \end{equation*}
        Moreover, we say $\alg$ is \emph{commutative} if the multiplication on $\alg$ is commutative. That is, for any $\alpha,\beta\in\alg$,
        \begin{equation*}
            \alpha\beta = \beta\alpha.
        \end{equation*}
    \end{definition}

    \begin{definition}{Unity}{of an Algebra}
        An element $\alpha\in\alg$ such that
        \begin{equation*}
            \forall\beta\in\alg \left[ \alpha\beta = \beta\alpha = \beta \right]  
        \end{equation*}
        is called the \emph{unity} of $\alg$. If such element exists, then it is unique, and we say $\alg$ is \emph{unital}.
    \end{definition}

    \begin{remark}
        The rest of this section will be devoted to the construction of the polynomial algebra. Recall that the set of all functions from $\N\cup \left\lbrace 0 \right\rbrace$ to $\F$ is a vector space, which we denote by $\F^\infty$. Then, any $f\in\F^\infty$ can be represented as an infinite sequence
        \begin{equation*}
            f = \left( f_0, f_1, \ldots \right) 
        \end{equation*}
        where $f_n = f(n)$ for each $n\in\N\cup \left\lbrace 0 \right\rbrace$.
    \end{remark}

    \begin{prop}{Algebra of Sequences}
        Let $\F^\infty$ be the algebra of all sequences in $\F$. That is, an element $f\in \F^\infty$ can be represented as
        \begin{equation*}
            f = \left( f_0, f_1, \ldots \right) \in \F^\infty, 
        \end{equation*}
        where $f: \N\cup \left\lbrace 0 \right\rbrace \to \F$ satisfy that $f_k = f(k)$ for any $k\in \N$. Define addition and scalar multiplication to be componentwise. Further define multiplication $\F^\infty\times\F^\infty\to\F^\infty$ to be such that
        \begin{equation*}
            \left( fg \right)_n = \sum^n_{i=0} f_ig_{n-i}.
        \end{equation*}
        Then $\F^\infty$ is an infinite-dimensional, unital, and commutative algebra over $\F$.
    \end{prop}

    \begin{proof}
        To verify that $\F^\infty$ is infinite-dimensional, observe that
        \begin{equation*}
            \left\lbrace e_k: k\in\N \right\rbrace \subseteq \F^\infty
        \end{equation*}
        is linearly independent. We verify other necessary properties componentwise. To verify distributavity, let $f = \left( f_0, f_1, \ldots \right), g = \left( g_0, g_1, \ldots \right), h = \left( h_0, h_1, \ldots \right)\in \F^\infty$. Then,
        \begin{equation*}
                \left( f(g+h) \right)_n = \sum^n_{i=0} f_i(g+h)_{n-i} = \sum^n_{i=0} f_i(g_{n-i} + h_{n-i}) = \sum^n_{i=0} f_ig_{n-i} + \sum^n_{i=0} f_ih_{n-i} = (fg)_n + (fh)_n. 
        \end{equation*}
        Similar proof holds for compatibility. To verify commutavity,
        \begin{equation*}
            (fg)_n = \sum^n_{i=0} f_ig_{n-i} = \sum^n_{i=0} g_if_{n-i} = (gf)_n.
        \end{equation*}
        We claim that
        \begin{equation*}
            1_{\F^\infty} = \left( 1, 0, \ldots \right) 
        \end{equation*}
        is the unity of $\F^\infty$. To verify this, observe that
        \begin{equation*}
            \left( 1_{\F^\infty}f \right)_n = \sum^n_{i=0} \left( 1_{\F^\infty} \right)_if_{n-i} = \sum^n_{i=0} \delta_{0i}f_{n-i} = f_n.
        \end{equation*}
        To verify the associativity, observe that
        \begin{flalign*}
            && \left[ (fg)h \right]_n & = \sum^n_{i=0} (fg)_ih_{n-i} = \sum^n_{i=0} \left( \sum^i_{j=0} f_jg_{i-j} \right)h_{n-i} = \sum^n_{i=0}\sum^i_{j=0} f_jg_{i-j}h_{n-i} && \\ 
                                   && & = \sum^n_{j=0}\sum^n_{i=j} f_jg_{i-j}h_{n-i} = \sum^n_{j=0} f_j\left( \sum^n_{i=j} g_{i-j}h_{n-k} \right) && \\
                                   && & = \sum^n_{j=0} f_j \left( \sum^{n-j}_{l=0} g_lh_{n-j-l} \right) = \sum^n_{j=0} f_j(gh)_{n-j} = \left[ f(gh) \right]_n,  && 
        \end{flalign*} 
        which is the desired result.
    \end{proof}

    \begin{remark}
        The vector
        \begin{equation*}
            x = \left( 0, 1, 0, \ldots \right)
        \end{equation*}
        plays a distinguished role in what follows, and we shall consistently denote it by $x$. The product of $x$ with itself $n$ times,
        \begin{equation*}
            x^n := \underbrace{x\cdot x\cdot\cdots\cdot x}_{n\text{ multiplicands}} = \underbrace{\left( 0, 0, \ldots, 0, 1, 0, \ldots \right)}_{n+1\text{th entry is }1}, 
        \end{equation*}
        shall be denoted by $x^n$. Notice that $x^0 = 1$ is the unity. 
    \end{remark}

    \section{Algebra of Polynomials}

    \begin{definition}{Algebra of Polynomials, Polynomial}{}
        We call the subspace of $\F^\infty$ spanned by $\left\lbrace 1, x, x^2, \ldots \right\rbrace$, denoted by $\F[x]$, the \emph{algebra of polynomials}. An element $f\in\F[x]$ of the algebra is called a \emph{polynomial}.
    \end{definition}

    \begin{remark}
        Notice that $f\in \F^\infty$ is a polynomial if there exists $a_0, a_1, \ldots, a_n\in \F$ with $a_n\neq 0$ such that
        \begin{equation*}
            f = a_0 1+ a_1x+ a_2x^2 + \cdots + a_nx^n,
        \end{equation*}
        for some $n\in \N\cup\left\lbrace 0 \right\rbrace $.
    \end{remark}

    \begin{remark}
        $\F[x]\subsetneq \F^\infty$. Moreover, $\F[x]$ is infinite-dimensional and the spanning set $\left\lbrace 1, x, x^2, \ldots \right\rbrace$ is linearly independent. That is, $\left\lbrace 1, x, x^2, \ldots \right\rbrace$ is a basis for $\F[x]$.
    \end{remark}

    \begin{definition}{Degree}{of a Polynomial}
        Let $f\in \F$ be nonzero. Write
        \begin{equation*}
            f = a_0 1 + a_1x + a_2x^2 + \cdots.
        \end{equation*}
        If $n\in \N$ is the largest element such that $a_n\neq 0$ (which clearly is unique), we say $n$ is the \emph{degree} of $f$ and denote as $\deg(f) = n$.
    \end{definition}

    \begin{remark}
        We do not assign a degree to $0\in \F[x]$.
    \end{remark}

    \begin{definition}{Coefficient, Leading Coefficient}{of a Polynomial}
        Let
        \begin{equation*}
            f = a_0 1+ a_1x + a_2x^2 + \cdots + a_nx^n\in \F[x].
        \end{equation*}
        Then we say $a_0, a_1, \ldots, a_n\in \F$ are \emph{coefficients} of $f$ and, in particular, $a_n$ with $n=\deg(f)$ is called the \emph{leading coefficient} of $f$.
    \end{definition}

    \begin{definition}{Scalar, Monic}{Polynomial}
        We say $f\in \F[x]$ is \emph{scalar} if $f=0$ or $\deg(f) = 0$. Moreover, we say $f\in\F[x]$ is \emph{monic} if the leading coefficient of $f$ is 1. 
    \end{definition}

    \begin{prop}{Properties of Polynomials}
        Let $f, g\in \F[x]$ be nonzero. Then the following holds.
        \begin{enumerate}
            \item $fg\neq 0$. In fact, $\deg(fg) = \deg(f)+\deg(g)$.
            \item $fg$ is monic if $f$ and $g$ are monic.
            \item $fg$ is scalar if $f$ and $g$ are scalar.
            \item $f+g = 0$ or $\deg(f+g)\leq \max \left( \deg(f), \deg(g) \right)$. Equality occurs whenever $\deg(f)\neq\deg(g)$.
        \end{enumerate}
    \end{prop}

    \begin{proof}
        To verify (a), let $n=\deg(f), m=\deg(g)$, and $k\in \N\cup \left\lbrace 0 \right\rbrace$. Observe that
        \begin{equation*}
            (fg)_{n+m+k} = \sum^{n+m+k}_{i=0} f_ig_{n+m+k-i},
        \end{equation*}
        where $f_i=0$ if $i>n$ and $g_{n+m+k-i}=0$ if $n+m+k-i>m\iff i<n+k$. That is, when $k\geq 1$, we have
        \begin{equation*}
            (fg)_{n+m+k} = 0.
        \end{equation*}
        When $k=0$,
        \begin{equation*}
            (fg)_{n+m+k} = (fg)_{n+m} = \sum^{n+m}_{i=0} f_ig_{n+m-i} = f_ng_m \neq 0.
        \end{equation*}
        Thus $\deg(fg) = n+m = \deg(f) + \deg(g)$. It follows that (b) and (c) are true as well. To verify (d), suppose that $\deg(f) = n \geq m = \deg(g)$ without loss of generality. Then for each $k\in \N$,
        \begin{equation*}
            (f+g)_{n+k} = f_{n+k} + g_{n+k} = 0 + 0 = 0. \eqedsym
        \end{equation*}
    \end{proof}

    \begin{cor}{$\F[x]$ Is Commutative and Unital}
        $\F[x]$ is a commutative, unital algebra.
    \end{cor}	

    \begin{proof}
        Observe that the associativity of $\F[x]$ is guaraneteed by the associativity of $\F^\infty$. Let $f,g\in \F$ with $\deg(f) = n$ and $\deg(g) = m$. Then $fg\in \F[x]$, where
        \begin{equation*}
            f = \sum^n_{i=0} a_ix^i \ \ \ \ g = \sum^m_{j=0} b_jx^j 
        \end{equation*}
        for some $a_0, a_1, \ldots, a_n, b_0, b_1, \ldots, b_m\in \F$. So
        \begin{equation*}
            fg = \sum^n_{i=0}\sum^m_{j=0} a_ib_jx^{i+j} = \sum^{n+m}_{p=0} \left( \sum^p_{l=0} a_lb_{p-l}x^p \right) = gf. \eqedsym
        \end{equation*}
    \end{proof}

    \begin{remark}
        (a) of Proposition 5.2 and Corollary 5.2.1 shows that $\F[x]$ (as a ring) is an integral domain. That is, $\F[x]$ is unital, commutative, and cancellative.
    \end{remark}

    \begin{remark}
        Let $\alg$ be a unital algebra over $\F$. Let $f\in \F[x]$. We evaluate $f$ on an element $\alpha\in \alg$ as follows. Denote
        \begin{equation*}
            \alpha^n = \underbrace{\alpha\cdot\alpha\cdot\cdots\cdot\alpha}_{n\text{ multiplicands}}
        \end{equation*}
        for each $n\in \N$, and $\alpha^0 = 1$. Moreover, write
        \begin{equation*}
            f = \sum^n_{i=0} c_ix^i\in \F[x],
        \end{equation*}
        where $c_0, c_1, \ldots, c_n\in\F$. We define $f(\alpha)\in\alg$ to be
        \begin{equation*}
            f(\alpha) := \sum^n_{i=0} c_i\alpha^i = c_0 1 + c_1\alpha + c_2\alpha^2 + \cdots + c_n\alpha^n.
        \end{equation*}
    \end{remark}

    \begin{prop}{Properties of Evaluating Polynomials on $\alpha\in\alg$}
        Let $f,g\in \F[x]$ and let $\alg$ be a unital linear algebra over $\F$. Let $\alpha\in \alg$ and $c\in\F$.
        \begin{enumerate}
            \item $(cf+g)(\alpha) = cf(\alpha)+g(\alpha)$.
            \item $(fg)(\alpha) = f(\alpha)g(\alpha)$.
        \end{enumerate}
    \end{prop}

    \begin{proof}
        (a) is a direct result of componentwise addition. To verify (b), write
        \begin{equation*}
            f = \sum^n_{i=0} f_ix^i \ \ \ \ g = \sum^m_{j=0} g_jx^j
        \end{equation*}
        where $n = \dim(f)$ and $m = \dim(g)$. Then,
        \begin{equation*}
            (fg)(\alpha) = \sum_{i,j} f_ig_j\alpha^{i+j} = \left( \sum^n_{i=0} f_i\alpha^i \right)\left( \sum^m_{j=0} g_j\alpha^j \right) = f(\alpha)g(\alpha). \eqedsym
        \end{equation*}
    \end{proof}

    \section{Lagrange Interpolation}

    \begin{remark}
        Let $n\in\N$. Throughout this section, we shall assume that $\F$ is a field with at least $n+1$ distinct elements $t_0,t_1,\ldots,t_n\in\F$. Moreover, we shall denote the subspace of $\F[x]$ containing polynomials of degree less than or equal to $n$, together with $0\in\F[x]$, by
        \begin{equation*}
            \F_n[x] = \left\lbrace f\in\F[x]:\deg(f)\leq n\lor f=0 \right\rbrace.
        \end{equation*}
    \end{remark}

    \begin{recall}{Linear Functional}{on a Vector Space}
        Let $V$ be a vector space over $\F$. A \emph{linear functional} (or \emph{linear form}) is a linear transformation $T:V\to \F$ from $V$ to its field of scalars $\F$.
    \end{recall}

    \continued{Remark}
    \noindent Define a function $L_i:\F_n[x]\to\F$ by
    \begin{equation*}
        f\mapsto f\left( t_i \right) 
    \end{equation*}
    for all $i\in\left\lbrace 0,1,\ldots,n \right\rbrace$. Then each $L_i$ is a linear functional on $\F_n[x]$. Now, what we intend to show here is that $\gamma = \left\lbrace L_0,L_1,\ldots,L_n \right\rbrace$ is a basis for $\F_n[x]^{*}$, the dual space of $\F_n[x]$. Of course, it is sufficient to show that there exists a basis $\beta = \left\lbrace p_0,p_1,\ldots,p_n \right\rbrace$ for $\F_n[x]^{*}$ such that $\gamma = \beta^{*}$. If such basis exists, then it is characterized by the fact that
    \begin{equation*}
        L_j\left( p_i \right) = p_i\left( t_j \right) = \delta_{ij}
    \end{equation*}
    for all $i\in\left\lbrace 0,1,\ldots,n \right\rbrace$. A way to construct such polynomials is by
    \begin{equation*}
        p_i = \prod^{n}_{j=0,j\neq i} \left( \frac{x-t_j}{t_i-t_j} \right) \in\F_n[x].
    \end{equation*}
    In particular, if $f\in\F_n[x]$ is $\sum^{n}_{i=0} c_ip_i$ for some $c_0,c_1,\ldots,c_n\in\F$, then for each $j\in\left\lbrace 0,1,\ldots,n \right\rbrace$,
    \begin{equation*}
        f\left( t_j \right) = \sum^{n}_{i=0} c_ip_i\left( t_j \right) = c_j.
    \end{equation*}
    This shows that $p_0,p_1,\ldots,p_n$ are linearly independent, since $0\in\F_n[x]$ has the property that $0(t)=0$ for any $t\in\F$. That is, if $f=0$, then $c_0=c_1=\cdots=c_n=0$. Clearly, $\left\lbrace 1,x,\ldots,x^n \right\rbrace$ is a basis for $\F_n[x]$, so $\dim\left( \F_n[x] \right) = n+1$. Thus $\beta$ is a basis for $\F_n[x]$, and, accordingly, $\gamma=\beta^{*}$ is also a basis for $\F_n[x]^{*}$.

    \begin{theorem}{Lagrange Interpolation}
        Let $t_0,t_1,\ldots,t_n\in\F$ be distinct. Define
        \begin{equation*}
            L_j(f) = f(t_j)
        \end{equation*}
        for each $j\in \left\lbrace 0, 1, \ldots, n \right\rbrace$. Then $\left\lbrace L_0, L_1, \ldots, L_n \right\rbrace$ is a basis for $\F_n[x]^*$.
    \end{theorem}

    \begin{cor}{Characterization of a Polynomial}
        Let $c_0, c_1, \ldots, c_n\in \F$. Then there exists a unique polynomial $f\in \F_n[x]$ such that
        \begin{equation*}
            f(t_i) = c_i.
        \end{equation*}
    \end{cor}	

    \begin{cor}{Lagrange's Interpolation Formula}
        Let $\F$ be a field with at least $n+1$ distinct elements and suppose we have constructed $p_0, p_1, \ldots, p_n$ as described in Remark 5.8. Then for each $f\in \F[x]$,
        \begin{equation*}
            f = \sum^{n}_{i=0} f(t_i)p_i.
        \end{equation*}
    \end{cor}	

    \begin{proof}
        Suppose
        \begin{equation*}
            f = \sum^{n}_{i=0} c_ip_i.
        \end{equation*}
        Then by Remark 5.8, $f\left( t_j \right) = c_j$. Thus
        \begin{equation*}
            f = \sum^{n}_{i=0} f(t_i)p_i. \eqedsym
        \end{equation*}
    \end{proof}

    \begin{definition}{Polynomial Function}{}
        Let $f\in \F[x]$ be a polynomial. We define a \emph{polynomial function} $\tilde{f}:\F\to\F$ by the mapping 
        \begin{equation*}
            t\mapsto f(t), 
        \end{equation*}
        where $f(t)$ is the evaluation of the polynomial $f$ at $t$.
    \end{definition}

    \begin{remark}
        By definition, every polynomial function is constructed in this way. However, it may happen that $\tilde{f}=\tilde{g}$ for two distinct polynomials $f,g\in \F$. Fortunately, this only occurs in the case where $\F$ is a finite field.
    \end{remark}

    \begin{remark}
        In order to describe in a precise way the relation between polynomials and polynomial functions, we proceed to define the product of two polynomial functions. Let $f,g\in\F[x]$ then we define the product of $\tilde{f}$ and $\tilde{g}$ by the mapping $t\mapsto \tilde{f}(t)\tilde{g}(t)$. Equvalently,
        \begin{equation*}
            \left( \tilde{f}\tilde{g} \right)(t) = \tilde{f}(t)\tilde{g}(t).
        \end{equation*}
        From Proposition 5.3, $(fg)(t) = f(t)g(t)$, and thus
        \begin{equation*}
            \widetilde{(fg)}(t) = \tilde{f}(t)\tilde{g}(t).
        \end{equation*}
        It follows that $\tilde{f}\tilde{g} = \widetilde{fg}$.
    \end{remark}

    \section{Polynomial Ideals}

    \begin{theorem}{Division Algorithm}
        Let $f,d\in \F[x]$ with $d\neq 0$. Then there exist unique polynomials $q,r\in \F[x]$ such that
        \begin{equation*}
            f = dq+r
        \end{equation*}
        where $r = 0$ or $\deg(r)<\deg(d)$.
    \end{theorem}

    \begin{lemma_inside}{}
        Let $f,d\in \F[x]$ be nonzero such that $\deg(f)\geq\deg(d)$. Then there exists $g\in \F[x]$ such that $f-dg = 0$ or $\deg(f-dg) < \deg(f)$.
    \end{lemma_inside}

    \begin{proof}
        Write
        \begin{equation*}
            f = \sum^n_{i=0} a_ix^i \ \ \ \ d = \sum^m_{j=0} b_jx^j
        \end{equation*}
        where $\deg(f) = n \geq m = \deg(d)$. So $a_n, b_m\neq 0$, which means
        \begin{equation*}
            g = \frac{a_n}{b_m}x^{n-m}
        \end{equation*}
        is a well-defined expression. Moreover,
        \begin{equation*}
            f - dg = \sum^n_{i=0} a_ix^i - \frac{a_n}{b_m}x^{n-m} \sum^m_{j=0} b_jx^j = a_nx^n - \left(\frac{a_n}{b_m}x^{n-m} b_mx^m\right) + \cdots = 0 + \cdots
        \end{equation*}
        by construction, so $\deg(f-dg) < \deg(f)$. 
    \end{proof}

    \ruleline{Proof of Theorem \thechapter.\thestcounter\ Begins Here}
    \begin{proof}
        When $\deg(f) < \deg(d)$, we have $q=0, r=f$ and their uniqueness is trivial. So suppose $\deg(f)\geq\deg(d)$. We verify the existence of $q, r\in \F[x]$ with $\deg(r)<\deg(d)$ or $r=0$ such that
        \begin{equation*}
            f = dq+r
        \end{equation*}
        first. By Lemma 5.5.1, there exists $q_1\in \F[x]$ such that
        \begin{equation*}
            \deg(f) > \deg(f-dq_1).
        \end{equation*}
        If $\deg(d) > \deg(f-dq_1)$ or $f-dq_1 = 0$, we are done. Otherwise, there exists another $q_2\in \F[x]$ such that
        \begin{equation*}
            \deg(f) > \deg(f-dq_1) > \deg(f-dq_1-dq_1).
        \end{equation*}
        Since this is a strict inequality, by continuing this process, we get polynomials $q_1, q_2, \ldots, q_n\in \F[x]$ such that
        \begin{equation*}
            \deg(d) > \deg(f-dq_1-dq_2-\cdots-dq_n)
        \end{equation*}
        or, if $\deg(d) = 0$,
        \begin{equation*}
            f - dq_1 - dq_2 - \cdots - dq_n = 0.
        \end{equation*}
        That is, $q = q_1+q_2+\cdots+q_n$ and $r = f-dq_1-dq_2-\cdots-dq_n$ satisfy the listed conditions. To verify uniqueness, suppose that there exist another $q'\in \F[x]$ such that
        \begin{equation*}
            f = dq'+r'
        \end{equation*}
        where $r' = 0$ or $\deg(r')<\deg(d)$. For the sake of contradiction, suppose that $q\neq q'$. Then $r-r' = (f-dq)-(f-dq') = dq'-dq \neq 0$ and we have
        \begin{equation*}
            \deg(r-r') = \deg(d)\deg(q'-q).
        \end{equation*}
        But clearly
        \begin{equation*}
            \deg(r-r') \leq \max(\deg(r), \deg(r')) < \deg(d),
        \end{equation*}
        so we have a contradiction. Thus $q'=q$ and, consequently, $r=r'$, which verifies the uniqueness.
    \end{proof}

    \begin{definition}{Divisor, Quotient, Remainder}{}
        Let $f, d, q, r\in F[x]$ satisfy conditions listed in Theorem 5.5. Then we call $d\neq 0$ the \emph{divisor}, $q$ the \emph{quotient}, and $r$ the \emph{remainder}.
    \end{definition}

    \begin{definition}{Divides}{}
        Let $f\in \F[x]$. We say $d\in \F[x]$ \emph{divides} $f$, denoted by $d\mid f$, if there exists $q\in\F[x]$ such that
        \begin{equation*}
            f = dq.
        \end{equation*}
    \end{definition}

    \begin{remark}
        Let $f\in\F[x]$ and suppose $d\mid f$ for some $d\in\F[x]$. Then Theorem 5.5 guarantees the existence and uniqueness of $q\in\F[x]$ such that $f = dq$.
    \end{remark}

    \begin{cor}{Remainder Theorem}
        Let $f\in \F[x]$ and $c\in \F$. Then
        \begin{equation*}
            (x-c)\mid f \iff f(c) = 0.
        \end{equation*}
    \end{cor}

    \begin{proof}
        By division algorithm,
        \begin{equation*}
            f = (x-c)q + r
        \end{equation*}
        for some unique $q,r\in \F[x]$. Notice that $r$ is a scalar, since if $r\neq 0$, then $\deg(r)< \deg(x-c) = 1$. That is,
        \begin{equation*}
            f(c) = (c-c)q(c)+r = r,
        \end{equation*}
        which means $f(c) = 0 \iff r = 0$. But $r=0$ exactly means $(x-c)\mid f$, as desired.
    \end{proof}

    \begin{definition}{Root}{of a Polynomial}
        Let $f\in \F[x]$. We say $c\in \F$ is a \emph{root} of $f$ if $f(c) = 0$.
    \end{definition}

    \clearpage
    \begin{cor}{Nonzero $f$ Has at Most $\deg(f)$ Roots}
        Let $f\in F[x]$ be nonzero and let $n=\deg(f)$. Then $f$ has at most $n$ roots.
    \end{cor}

    \begin{proof}
        We proceed by induction. When $\deg(f) = 0$, $f = a_0 1$ so $f\neq 0$ for all $c\in \F$. When $\deg(f) = 1, f = a_0 1+a_1x$. That is
        \begin{equation*}
            f(c) = 0 \iff c = -\frac{a_1}{a_0}.
        \end{equation*}
        Now suppose that every $f\in \F[x]$ with $\deg(f) = k$ has at most $k$ roots. Let $g\in \F[x]$ be such that $\deg(g) = k+1$. If $g$ does not have any root, then we are done. So suppose $g$ has a root $c\in \F$. Then by the remainder theorem,
        \begin{equation*}
            g = (x-c)q
        \end{equation*}
        for some unique $q\in \F[x]$ with $\deg(q) = k$. But by induction hypothesis, $q$ has at most $k$ roots, so $g$ has at most $k+1$ roots, as desired.
    \end{proof}

    \begin{remark}
        If $\deg(f)>1$, there needs not exist any roots.
    \end{remark}

    \begin{definition}{Algebraically Closed}{Field}
        Let $\F$ be a field. We say $\F$ is \emph{algebraically closed} if every polynomial $f\in\F[x]$ over $\F$ has a root.
    \end{definition}

    \begin{definition}{Formal Derivative}{of a Polynomial}
        Define linear operator $D: \F[x]\to \F[x]$ by
        \begin{equation*}
            \sum^n_{i=0} a_ix^i\mapsto \sum^n_{i=1} ia_ix^{i-1} 
        \end{equation*}
        For any $f\in\F[x]$, we call $Df$ the \emph{formal derivative} of $f$.
    \end{definition}

    \begin{theorem}{Binomial Theorem}
        Let $\alg$ be an commutative algebra over $\F$ with characteristic zero and let $\alpha, \beta \in \alg$. Then
        \begin{equation*}
            (\alpha + \beta)^n = \sum^n_{r=0} \binom{n}{r} \alpha^{n-r}\beta^r.
        \end{equation*}
    \end{theorem}

    \begin{proof}
        We proceed inductively. Observe that
        \begin{equation*}
            (\alpha+\beta)^1 = \alpha + \beta = \binom{1}{0} \alpha^1\beta^0 + \binom{1}{1}\alpha^0\beta^1 = \sum^1_{k=0} \binom{1}{k} \alpha^{1-k}\beta^k.
        \end{equation*}
        Now suppose
        \begin{equation*}
            (\alpha+\beta)^k = \sum^k_{r=0} \binom{k}{r} \alpha^{k-r}\beta^r.
        \end{equation*}
        Then,
        \begin{equation*}
            (\alpha+\beta)^{k+1} = (\alpha+\beta) \sum^k_{r=0} \binom{k}{r} \alpha^{k-r}\beta^r = \sum^k_{r=0} \binom{k}{r} \alpha^{k+1-r}\beta^r + \sum^k_{r=0} \binom{k}{r} \alpha^{k-r}\beta^{r+1}.
        \end{equation*}
        Since $\F$ has zero characteristic, the coefficient of the term $\alpha^{k+1-r}\beta^r$ is given by the expression
        \begin{equation*}
            \binom{k}{r} + \binom{k}{r-1} = \binom{k+1}{r},
        \end{equation*}
        which exactly means
        \begin{equation*}
            \sum^k_{r=0} \binom{k}{r} \alpha^{k+1-r}\beta^r + \sum^k_{r=0} \binom{k}{r} \alpha^{k-r}\beta^{r+1} = \sum^{k+1}_{r=0} \binom{k+1}{r} \alpha^{k+1-r}\beta^r.
        \end{equation*}
        So by the principle of mathematical induction,
        \begin{equation*}
            (\alpha+\beta)^n = \sum^n_{r=0} \binom{n}{r} \alpha^{n-r}\beta^r,
        \end{equation*}
        as desired.
    \end{proof}

    \begin{theorem}{Taylor's Formula}
        Let $\F$ be a field with characteristic zero and let $f\in \F[x]$ be a nonzero polynomial with $\deg(f)\leq n$ for some $n\in \N$. Then,
        \begin{equation*}
            f = \sum^{n}_{k=0} \frac{D^kf}{k!} (c)(x-c)^k
        \end{equation*}
        for any $c\in \F$.
    \end{theorem}

    \begin{proof}
        We first verify the result for the standard basis vectors $1, x, x^2, \ldots\in\F[x]$. Let $f = x^m$ for some $m\leq n$. Then,
        \begin{flalign*}
            && \sum^{n}_{k=0} \frac{D^kf}{k!}(c)(x-c)^k & = \sum^{n}_{k=n+1} \frac{D^kf}{k!}(c)(x-c)^k + \sum^{m}_{k=0} \frac{D^kf}{k!} (c)(x-c)^k && \\
            && & = \sum^{m}_{k=0} \frac{\frac{m!}{(m-k)!}c^{m-k}}{k!} (x-c)^k = \sum^{m}_{k=0} \binom{m}{k} c^{m-k} (x-c)^k = (c+(x-c))^m = x^m. && 
        \end{flalign*} 
        That is, for any $f = \sum^{m}_{p=0} a_px^p$ for some $m\leq n$, 
        \begin{flalign*}
            && \sum^{n}_{k=0} \frac{D^kf}{k!}(c)(x-c)^k & = \sum^{n}_{k=0} \frac{D^k \left( \sum^{m}_{p=0} a_px^p \right) }{k!} (c)(x-c)^k = \sum^{n}_{k=0} \sum^{m}_{p=0}  \frac{a_p D^k (x^p)}{k!} (c)(x-c)^k && \\
            && & = \sum^{m}_{p=0} a_p \sum^{n}_{k=0} \frac{D^k(x^p)}{k!} (c)(x-c)^k = \sum^{m}_{p=0} a_px^p = f, && 
        \end{flalign*} 
        as desired.
    \end{proof}

    \begin{definition}{Multiplicity}{of a Root}
        Let $f\in \F[x]$ and $c\in \F$. We say $m\in \N$ is the \emph{multiplicity} of $c$ provided that $(x-c)^m\mid f$ and $(x-c)^{m+1}\nmid f$.
    \end{definition}

    \begin{prop}{}
        Let $\F$ be a field with characteristic zero and $f\in \F[x]$. Then $c\in \F$ is a root with multiplicity $r$ if and only if $D^if(c)=0$ for each $i\in\left\lbrace 0,1,\ldots,r-1 \right\rbrace$ and $D^rf=0$.
    \end{prop}

    \begin{proof}
        For the forward direction, suppose $c\in \F$ is a root of multiplicity $r$. Write $f = (x-c)^rq$ where $\deg(f) = n$, $\deg(q) = n-r$, and $q(c)\neq 0$. By Taylor's formula,
        \begin{equation*}
            f = (x-c)^rq = (x-c)^r \sum^{n-r}_{k=0} \frac{D^kq}{k!} (c)(x-c)^k = \sum^{n-r}_{k=0} \frac{D^kq}{k!}(c)(x-c)^{k+r}. 
        \end{equation*}
        Since $\left\lbrace 1, (x-c), (x-c)^2, \ldots \right\rbrace$ is a basis for $\F[x]$, the above expression is the unique representation of $f$ as a linear combination of $1, (x-c), (x-c)^2, \ldots, (x-c)^n$. So
        \begin{equation*}
            \sum^{n}_{k=0} \frac{D^kf}{k!}(c)(x-c)^k = \sum^{n-r}_{k=0} \frac{D^kq}{k!} (c)(x-c)^{k+r} = \sum^{n}_{k=r}  \frac{D^{k-r}q}{(k-r)!} (c)(x-c)^k. 
        \end{equation*}
        It follows that $D^rf(c) = r! q(c) \neq 0$ and $\frac{D^kf}{k!}(c) = 0$ for each $k\in \left\lbrace 0, 1, \ldots, r-1 \right\rbrace$. For the reverse direction, suppose that $D^k f(c) = 0$ for each $K\in \left\lbrace 0, 1, \ldots, r-1 \right\rbrace$ and $D^rf(c) \neq 0$. Then
        \begin{equation*}
            f = \sum^{n}_{k=0} \frac{D^kf}{k!}(c)(x-c)^k = \sum^{n}_{k=r}  \frac{D^kf}{k!}(c)(x-c)^k = (x-c)^r \sum^{n}_{k=r} \frac{D^kf}{k!} (c)(x-c)^{k-r}
        \end{equation*}
        so $(x-c)^r\mid f$ but $(x-c)^{r+1}\nmid f$, as desired.
    \end{proof}

    \begin{definition}{Ideal}{of a Polynomial Space}
        We say a subspace $M\subseteq \F[x]$ is an \emph{ideal} of $\F[x]$ if
        \begin{equation*}
            f\in M \land g\in \F[x] \implies fg\in M.
        \end{equation*}
    \end{definition}

    \begin{theorem}{Ideal Test}
        Let $M\subseteq \F[x]$. Then $M$ is an ideal of $\F[x]$ if the following holds.
        \begin{enumerate}
            \item For each $f, g\in M$, $f-g\in M$.
            \item For each $f\in M$ and $h\in \F[x]$, $fh = hf \in M$.
        \end{enumerate}
    \end{theorem}

    \begin{definition}{Generator}{of an Ideal}
        Let $M\subseteq \F[x]$ be an ideal. We say $\left\lbrace d_1, d_2, \ldots, d_n \right\rbrace \subseteq \F[x]$ is a generator of $M$ if
        \begin{equation*}
            M = \left\lbrace \sum^{n}_{i=1} d_if : f\in \F[x] \right\rbrace.
        \end{equation*}
        For such case, we say $M$ is the ideal \emph{generated} by $\left\lbrace d_1, d_2, \ldots, d_n \right\rbrace$ and we often denote $M$ by $\left< d_1, d_2, \ldots, d_n \right>$.
    \end{definition}

    \begin{definition}{Principal Ideal}{}
        Let $d\in \F[x]$. We say $\left<d\right>$ is the \emph{principal ideal} generated by $d$ of $\F[x]$.
    \end{definition}

    \begin{definition}{Trivial Ideal}{}
        For any polynomial space $\F[x]$, $\left\lbrace 0 \right\rbrace$ is an ideal, which we call the \emph{trivial ideal}.
    \end{definition}

    \begin{prop}{$\F[x]$ Is a Principal Idean Domain}
        Let $M\subseteq \F[x]$ be a nontrivial ideal. Then $M = \left<d\right>$ for some monic $d\in \F[x]$.
    \end{prop}

    \begin{proof}
        Let $d\in M$ be a monic polynomial with minimum degree. We claim that $\left< d \right> = M$. Clearly $\left< d \right> \subseteq M$. Let $f\in M$ be nonzero. By division algorithm,
        \begin{equation*}
            f = dq+r
        \end{equation*}
        for some $q,r\in \F[x]$ with $\deg(d) > \deg(r)$ or $r = 0$. By the closure under subtraction of vector spaces, $r = f-dq\in M$. So if $r\leq 0$, we violate the minimality of $d$, since $\deg(d) > \deg(r)$. That is, $f = dq$ for some $q\in \F[x]$, so $f\in \left< d \right>$, as desired. 
    \end{proof}

    \begin{cor}{Uniqueness of the Monic Generator}
        Let $p_1, p_2, \ldots, p_k\in \F[x]$ be nonzero. Then there exists a unique monic $d\in \F[x]$ that satisfies
        \begin{enumerate}
            \item $d\in \left< p_1, p_2, \ldots, p_k \right>$.  
            \item $d\mid p_1\land d\mid p_2\land \cdots \land d\mid p_k$.
            \item $\forall f\in \F[x] \left[ f\mid p_1\land f\mid p_2\land \cdots f\mid p_k \implies f\mid d \right]$. 
        \end{enumerate}
    \end{cor}	

    \begin{proof}
        We claim that the monic generator $d\in \F[x]$ of $\left< p_1, p_2, \ldots, p_k \right>$ uniquely satisfies (a), (b), and (c). Notice that $\left< d \right> = \left< p_1, p_2, \ldots, p_k \right>$ means $d\in \left< p_1, p_2, \ldots, p_k \right>$ and (b) is a direct result of Proposition 5.10. To verify that $d$ satisfies (c), observe that $d = \sum^{k}_{i=1} p_if_i$ for some $f_1, f_2, \ldots, f_k\in \F[x]$ and, for each $i\in \left\lbrace 1, 2, \ldots, k \right\rbrace
        $, $p_i = fq_i$ for some $q_i\in \F[x]$. So,
        \begin{equation*}
            d = \sum^{k}_{i=1} p_if_i = \sum^{k}_{i=1}  fq_if_i = f \sum^{k}_{i=1} q_if_i
        \end{equation*}
        is divisible by $f$. To verify the uniqueness, suppose $d'\in\F[x]$ satisfies (a), (b), and (c). Then by (b), $d'\mid p_1\land d'\mid p_2\land\cdots\land d'\mid p_k$, so $d'\mid d$ by (c). But by symmetry $d\mid d'$ as well. Since both $d$ and $d'$ are monic, it follows that $d=d'$, as required.
    \end{proof}

    \begin{definition}{Greatest Common Divisor}{of Polynomials}
        Let $p_1, p_2, \ldots, p_k\in \F[x]$ be nonzero. We call the unique monic generator $d\in \F[x]$ of $\left< p_1, p_2, \ldots, p_k \right>$ the \emph{greatest common divisor} of $p_1, p_2, \ldots, p_k$, denoted as $d = \gcd \left( p_1, p_2, \ldots, p_k \right)$. 
    \end{definition}

    \begin{definition}{Coprime}{Polynomials}
        We say nonzero $p_1, p_2, \ldots, p_k$ are \emph{coprime} if $\gcd(p_1, p_2, \ldots, p_k) = 1$.
    \end{definition}

    \begin{prop}{Alternative Definitions of Coprime Polynomials}
        Let $p_1, p_2, \ldots, p_k\in \F[x]$. Then the following are equivalent.
        \begin{enumerate}
            \item $p_1, p_2, \ldots, p_k$ are coprime.
            \item There exists $q_1, q_2, \ldots, q_k\in F[x]$ such that $\sum^k_{i=1} p_iq_i = 1$.
            \item $\left< p_1, p_2, \ldots, p_k \right> = \F[x]$. 
        \end{enumerate}
    \end{prop}

    \begin{theorem}{Bezout's Lemma for Polynomials}
        Let $M\subseteq \F[x]$ be an ideal. For each $p,q\in M$, $\gcd(p,q) = d$ if and only if $d\mid p$, $d\mid q$, and $\exists f,g\in \F[x]$ such that $d = fp+gq$.
    \end{theorem}

    \begin{definition}{Reducible, Irreducible, Prime}{Polynomials}
        Let $f\in \F[x]$. We say $f$ is \emph{reducible} if there exist $g,h\in \F[x]$ such that $\deg(g), \deg(h)\geq 1$ and $f=gh$. Otherwise, we say $f$ is \emph{irreducible}. An irreducible $f$ is \emph{prime} if $\deg(f)\geq 1$.
    \end{definition}

    \begin{remark}
        Every polynomial with degree 1 is prime.
    \end{remark}

    \begin{prop}{Alternative Definition of Prime}
        Let $f,g,h\in \F[x]$. If $f$ is prime and $f\mid gh$, then $f\mid g$ or $f\mid h$.
    \end{prop}

    \begin{proof}
        Without loss of generality, suppose $f$ is monic. Let $d=\gcd(g,h)$. Then $d$ is a monic polynomial that divides $f$, so $d=f$ or $d=1$. Since if $d=f$, then $f\mid g$ and $f\mid h$, suppose $d=1$. It follows that $f$ is relatively prime with $g$ or $h$, so without loss of generality, suppose $f$ is relatively prime with $g$. Then there are $f', g'\in \F[x]$ such that $ff'+gg'=1$. By multiplying both sides by $h$< we get
        \begin{equation*}
            h = ff'h + gg'h = f(f'h) + (gh)g'.
        \end{equation*}
        Clearly $f\mid f(f'h)$ and $f\mid gh$. That is, $f\mid h$, as required.
    \end{proof}

    \begin{cor}{If $f$ Is Prime, Then $f\mid \prod^n_{i=1} p_i \implies f\mid p_k$ for Some $k\in \left\lbrace 1, 2, \ldots, n \right\rbrace $}
        Let $f, p_1, p_2, \ldots, p_n\in \F[x]$ be such that $f$ is prime and $f\mid \prod^n_{i=1} p_i$. Then there is $k\in \left\lbrace 1, 2, \ldots, n \right\rbrace$ such that $f\mid p_k$.
    \end{cor}	

    \begin{proof}
        We proceed inductively. When $n=1$ the result holds trivially. Suppose that the result holds for $n=m$ and $f\mid \prod^n_{i=1} p_i$. Then by Proposition 5.13,
        \begin{equation*}
            f\left|\prod^n_{i=1} p_i \lor f\mid p_{m+1}\right.
        \end{equation*}
        If $f\mid p_{m+1}$, then $k=m+1$ and we are done. If $f\mid\prod^n_{i=1} p_i$, then the result follows by the induction hypothesis.
    \end{proof}

    \clearpage
    \begin{theorem}{Unique Factorization of Monic Polynomials}
        Let $f\in \F[x]$ be nonscalar and monic. Then there exist unique monic, prime $p_1, p_2, \ldots, p_n\in \F[x]$ such that $f = \prod^n_{i=1} p_i$.
    \end{theorem}

    \begin{proof}
        We proceed inductively. When $\deg(f)=1$, $f$ itself is prime. For some $k\in \N$, suppose the result for all $f\in \F[x]$ with $\deg(f)\leq k$. Let $g\in \F[x]$ be monic with $\deg(g) = k+1$. If $g$ is prime, then we are done. So suppose that $g$ is not prime. Then there are monic $p, q\in \F[x]$ with $\deg(p),\deg(q)\geq 1$ such that $g=pq$. Clearly $\deg(p), \deg(q)\leq m$, so there exist unique primes $p_1, p_2, \ldots, p_r, p_{r+1}, \ldots, p_n$ such that $p =
        \prod^{r}_{i=1} p_i$ and $q = \prod^{n}_{i=r+1} p_i$. That is, $g = \prod^{n}_{i=1} p_i$. To verify uniqueness, suppose that there exist nomic, prime $q_1, q_2, \ldots, q_n\in \F[x]$ such that $g = \prod^{h}_{j=1} q_j$. Then for each $i\in \left\lbrace 1, 2, \ldots, n \right\rbrace$,
        \begin{equation*}
            p_i \left| \prod^{h}_{j=1} q_j\right. 
        \end{equation*}
        so $p_i\mid q_j$ for some $j\in \left\lbrace 1, 2, \ldots, h \right\rbrace$. But both $p_i$ and $q_j$ are prime and monic, which means $p_i = q_j$. By symmetry, for each $j\in \left\lbrace 1, 2, \ldots, r \right\rbrace$, $q_j = p_i$ for some $i\in \left\lbrace 1, 2, \ldots, n \right\rbrace$. It follows that $n = r$, and by some renumbering,
        \begin{equation*}
            \forall i\in \left\lbrace 1, 2, \ldots, n \right\rbrace \left[ p_i = q_j \right],
        \end{equation*}
        as desired.
    \end{proof}

    \begin{remark}
        Let $f\in\F[x]$ be a nonscalar monic polynomial. Then Theorem 5.14 guarantees a unique representation of $f$ in terms of products of monic prime polynomials, some of which may be repeated. In other words, if $p_1,p_2,\ldots,p_k\in F[x]$ are the distinct primes which occur in the factorization of $f$, then
        \begin{equation*}
            f = \prod^{k}_{i=1} p_i^{n_i}
        \end{equation*}
        for some $n_1,n_2,\ldots,n_k\in\N$. 
    \end{remark}

    \begin{definition}{Primary Decomposition}{of a Monic Polynomial}
        Consider Remark 5.14. We call
        \begin{equation*}
            f = \prod^{k}_{i=1} p_i^{n_i}
        \end{equation*}
        the \emph{primary decomposition} of $f$.
    \end{definition}

    \begin{prop}{}
        Let $f\in\F[x]$ be a nonscalar monic polynomial and let
        \begin{equation*}
            f = \prod^{k}_{i=1} p_i^{n_i}
        \end{equation*}
        be the primary decomposition of $f$. For each $j\in\left\lbrace 1,2,\ldots,k \right\rbrace$, let
        \begin{equation*}
            f_j = \frac{f}{p_j^{n_j}} = \prod^{k}_{i=1,i\neq j} p_i^{n_i} .
        \end{equation*}
        Then $f_1,f_2,\ldots,f_n$ are coprime.
    \end{prop}

    \begin{proof}
        For the sake of contradiction, suppose that there exists prime $p\in\F[x]$ such that
        \begin{equation*}
            p\mid f_i
        \end{equation*}
        for each $i\in\left\lbrace 1,2,\ldots,k \right\rbrace$. It is clear that $p=p_j$ for some $j\in\left\lbrace 1,2,\ldots,k \right\rbrace$ by Theorem 5.14. But this is a contradiction, since clearly
        \begin{equation*}
            p_j\nmid f_j. \eqedsym
        \end{equation*}
    \end{proof}

    \clearpage
    \begin{prop}{$f$ Is a Product of Distinct Primes If and Only If $f$ and $Df$ Are Coprime}
        Let $f\in \F[x]$. Then $f$ is a product of distinct primes if and only if $\gcd(f, Df) = 1$.
    \end{prop}

    \begin{proof}
        For the forward direction, suppose that there exist distinct primes $p_1, p_2, \ldots, p_n\in \F[x]$ such that $f = \prod^{n}_{i=1} p_i$. By product rule,
        \begin{equation*}
            Df = \sum^{n}_{j=1} D(p_j) \prod^{n}_{i=1, i\neq j} p_i. 
        \end{equation*}
        Clearly $p_j\nmid Dp_j$ since $\deg(p_j) > \deg(Dp_j)$. That is, for each $j\in \left\lbrace 1, 2, \ldots, n \right\rbrace$, $p_j\nmid Df$, or, $f$ and $Df$ are coprime. For the reverse direction, suppose that there exists a prime $p_j\in \F[x]$ such that $f = p_j^k \prod^{n}_{i=1, i\neq j} p_i$ for some $k\geq 2$. By product rule and chain rule,
        \begin{equation*}
            Df = \left( \sum^{n}_{s=1, s\neq j} D(p_s) \prod^{n}_{i=1, i\neq s} p_i  \right) + kp_j^{k-1} D(p_j) \prod^{n}_{i=1, i\neq j} p_i, 
        \end{equation*}
        where $p_j$ divides both summands. So $\gcd(f, Df)\neq 1$, as desired.
    \end{proof}

    \begin{definition}{Algebraically Closed}{Field}
        Let $\F$ be a field. We say $\F$ is \emph{algebraically closed} if every nonscalar $f\in \F[x]$ has a root $c\in \F$.
    \end{definition}

    \begin{prop}{Alternative Definitions of Algebraic Closure}
        Let $\F$ be a field. Then the following are equivalent.
        \begin{enumerate}
            \item $\F$ is algebraically closed.
            \item For each $f\in \F[x]$, 
                \begin{equation*}
                    f = c\prod^{n}_{i=1} \left( x-c_i \right) ^{p_i}
                \end{equation*}
                for some $c, c_1, c_2, \ldots, c_n\in \F$ and $p_1, p_2, \ldots, p_n\in \N$.
            \item Every prime polynomial of $\F[x]$ has degree 1.
        \end{enumerate}
    \end{prop}

\end{document}
